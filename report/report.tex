% CVPR 2022 Paper Template
% based on the CVPR template provided by Ming-Ming Cheng (https://github.com/MCG-NKU/CVPR_Template)
% modified and extended by Stefan Roth (stefan.roth@NOSPAMtu-darmstadt.de)

\documentclass[10pt,twocolumn,letterpaper]{article}

%%%%%%%%% PAPER TYPE  - PLEASE UPDATE FOR FINAL VERSION
%\usepackage[review]{cvpr}      % To produce the REVIEW version
\usepackage{cvpr}              % To produce the CAMERA-READY version
%\usepackage[pagenumbers]{cvpr} % To force page numbers, e.g. for an arXiv version

% Include other packages here, before hyperref.
\usepackage{graphicx}
\usepackage{amsmath}
\usepackage{amssymb}
\usepackage{booktabs}
\usepackage{caption}
\usepackage{subcaption}

\usepackage[pagebackref,breaklinks,colorlinks]{hyperref}


% Support for easy cross-referencing
\usepackage[capitalize]{cleveref}
\crefname{section}{Sec.}{Secs.}
\Crefname{section}{Section}{Sections}
\Crefname{table}{Table}{Tables}
\crefname{table}{Tab.}{Tabs.}


%%%%%%%%% PAPER ID  - PLEASE UPDATE
\def\cvprPaperID{***} % *** Enter the CVPR Paper ID here
\def\confName{ x}
\def\confYear{2022}


\begin{document}

%%%%%%%%% TITLE - PLEASE UPDATE
\title{Report}

\author{Matteo Gambino\\
s287572
\and
Michele Pierro\\
s287846
\and
Fabio Grillo\\
s287873
}
\maketitle

%%%%%%%%% ABSTRACT
\begin{abstract}
   In order to predict the location of a query image by retrieving annotated photographs with 
   similar descriptors needs an efficient and reliable generation of those descriptors. 
   In order to accomplish that objective, is fundamental that the network focuses on portion
   of the various images that contains useful information and at the same time ignore not 
   informative areas like the ones containing elements like cars or pedestrians. For that 
   reason attention layers are fundamental in the proposed network. In addition to that we 
   are comparing state of the art techniques for the visual geolocalization task like GEM \cite{GEM}
   and NetVLAD \cite{NETVLAD}.
\end{abstract}

%%%%%%%%% BODY TEXT
\section{Introduction}

\section{Related works}
\section{Methods}
\section{Experiments}
 \begin{tabular}{|l|l|l|l|l|}
\hline
          & R@1       &  R@2    & R@10 & R@20      \\
\hline
lr = 1e-3            & 81.9    & 91.8 & 94.8 & 96.8          \\
\hline
lr = 1e-4            & 82.2    & 93.0 & \textbf{95.4} & \textbf{97.1}          \\
\hline
lr = 1e-5            & \textbf{83.5}    & \textbf{93.1} & 94.3 & \textbf{97.1}          \\
\hline
\end{tabular}
\section{Ablation study}
\section{Conclusions}

%%%%%%%%% REFERENCES
{\small
\bibliographystyle{ieee_fullname}
\bibliography{egbib}
}

\end{document}
