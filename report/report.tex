% CVPR 2022 Paper Template
% based on the CVPR template provided by Ming-Ming Cheng (https://github.com/MCG-NKU/CVPR_Template)
% modified and extended by Stefan Roth (stefan.roth@NOSPAMtu-darmstadt.de)

\documentclass[10pt,twocolumn,letterpaper]{article}

%%%%%%%%% PAPER TYPE  - PLEASE UPDATE FOR FINAL VERSION
%\usepackage[review]{cvpr}      % To produce the REVIEW version
\usepackage{cvpr}              % To produce the CAMERA-READY version
%\usepackage[pagenumbers]{cvpr} % To force page numbers, e.g. for an arXiv version

% Include other packages here, before hyperref.
\usepackage{graphicx}
\usepackage{amsmath}
\usepackage{amssymb}
\usepackage{booktabs}
\usepackage{caption}
\usepackage{subcaption}

\usepackage[pagebackref,breaklinks,colorlinks]{hyperref}


% Support for easy cross-referencing
\usepackage[capitalize]{cleveref}
\crefname{section}{Sec.}{Secs.}
\Crefname{section}{Section}{Sections}
\Crefname{table}{Table}{Tables}
\crefname{table}{Tab.}{Tabs.}


%%%%%%%%% PAPER ID  - PLEASE UPDATE
\def\cvprPaperID{***} % *** Enter the CVPR Paper ID here
\def\confName{ x}
\def\confYear{2022}


\begin{document}

%%%%%%%%% TITLE - PLEASE UPDATE
\title{Report}

\author{Matteo Gambino\\
s287572
\and
Michele Pierro\\
s287846
\and
Fabio Grillo\\
s287873
}
\maketitle

%%%%%%%%% ABSTRACT
\begin{abstract}
   In order to predict the location of a query image by retrieving annotated photographs with 
   similar descriptors needs an efficient and reliable generation of those descriptors. 
   In order to accomplish that objective, is fundamental that the network focuses on portion
   of the various images that contains useful information and at the same time ignore not 
   informative areas like the ones containing elements like cars or pedestrians. For that 
   reason attention layers are fundamental in the proposed network. In addition to that we 
   are comparing state of the art techniques for the visual geolocalization task like GeM \cite{GEM}
   and NetVLAD \cite{NETVLAD}.
\end{abstract}

%%%%%%%%% BODY TEXT
\section{Introduction}

\section{Related works}
\section{Methods}
Like \cite{GEM}, \cite{NETVLAD}, \cite{CRN} we have casted the problem of place recognition as the task of image
retrieval. We have implemented 3 different networks all based on the ResNet-18 \cite{resnet} backbone
without the fully connected layers and the last convolution layer. On the top of this backbone we 
have inserted 4 different heads, inspired by the works of \cite{GEM}, \cite{NETVLAD}, \cite{CRN}, in order
to generate the image descriptors. 
\paragraph*{Base Head}
This is the simplest head we have used and it's necessary in order to have a baseline to compare the other
results. After the last convolution layer of ResNet-18 we have normalized the feature map and used average 
pooling in order to generate the descriptors. This simple head tries to extract from the query the spatial
information by comparing the average value of the features in a given area and represent the traditional way 
to extract those descriptors.
\paragraph*{GeM head}
Following the work of \cite{GEM}, we have used a Generalized Mean approach in order to extract better 
descriptors for the query image. The generalized mean we are using is defined as:
\begin{equation}
   f_k = ({{1}\over{X_k}} \sum_{x \in X_k} x^{p_k} ) ^ {1\over{p_k}}
\end{equation}
where $X_k $ represent one of the normalized features map and $p_k $ is the pooling parameter. This 
pooling parameter is expressing how much is localized the zone of the image the network is focusing on.
The $p_k $ parameter, although it can be learned and inserted into back propagation, it has been fixed
and a single value is used for each activation map as suggested by \cite{GEM}. We have
inserted a fully connected layer that takes as input the pooled features in order to whiten the image
descriptors since it has been shown by \cite{GEM} that this approach is providing better results than
using other strategies like PCA.
\paragraph*{NetVLAD head}
\paragraph*{CRN head}
Seen the results provided from the previous implemented heads and the success of the attention layers to make 
a network focus on relevant only parts of an image, we have decided to add an attention layer at the NetVLAD
head following the approach proposed by \cite{CRN}. This is perfectly integreated in the NetVlad architecture and
it has the duty to produce a map that rescales the weights produced by the soft assignment step of the NetVLAD layer.
This layer is composed by an initial average pooling sub-layer
that has the duty to reduce the dimensionality of the feature maps produced by the backbone. 
Differently from what specified in \cite{CRN}, it is not reducing the features maps to a fixed 
size but it is simply reducing by a half the dimensions of the features. In order to capture features at different
spatial resolutions, 3 convolution filters (with kernel sizes respectively of 3, 5, 7) are applied to the pooled features.
\section{Experiments}
 \begin{tabular}{|l|l|l|l|l|}
\hline
          & R@1       &  R@2    & R@10 & R@20      \\
\hline
lr = 1e-3            & 81.9    & 91.8 & 94.8 & 96.8          \\
\hline
lr = 1e-4            & 82.2    & 93.0 & \textbf{95.4} & \textbf{97.1}          \\
\hline
lr = 1e-5            & \textbf{83.5}    & \textbf{93.1} & 94.3 & \textbf{97.1}          \\
\hline
\end{tabular}
\section{Ablation study}
\section{Conclusions}

%%%%%%%%% REFERENCES
{\small
\bibliographystyle{ieee_fullname}
\bibliography{egbib}
}

\end{document}
